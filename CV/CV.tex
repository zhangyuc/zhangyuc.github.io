\documentclass{res}
\setlength{\textheight}{8.6in} % increase text height to fit resume on 1 page
\newsectionwidth{0pt}  % So the text is not indented under section headings
\usepackage[colorlinks,urlcolor=blue,linkcolor=blue,anchorcolor=blue,citecolor=blue]{hyperref}
\usepackage{enumitem}

\newenvironment{my_item}{
\begin{itemize}
  \setlength{\itemsep}{0pt}
  \setlength{\parskip}{0pt}
  \setlength{\parsep}{0pt}}
{\end{itemize}
}

\begin{document}

% Center the name over the entire width of resume:
 \moveleft.5\hoffset\centerline{\LARGE\bf Yuchen Zhang}
% Draw a horizontal line the whole width of resume:
 \moveleft\hoffset\vbox{\hrule width\resumewidth height 1pt}\smallskip
% address begins here
\address{\bf\small  CONTACT\\\url{https://zhangyuc.github.io/}\\
 Email: zhangyuc@cs.stanford.edu\\ Phone: (+1)-510-423-1353}
\address{\bf\small  ADDRESS\\Gates Computer Science 254,\\Stanford University,\\Stanford, CA 94305.}

\begin{resume}

{\Large\bf Education}

\vspace{-5pt}
\textbf{University of California, Berkeley}\hfill\textbf{2011 - 2016}\\
Doctor of Philosophy in Computer Science\\
{Advised by Michael I. Jordan and Martin J. Wainwright}

\vspace{-5pt}
\textbf{University of California, Berkeley} \hfill\textbf{2011 - 2013}\\
Master of Arts in Statistics

\vspace{-5pt}
\textbf{Tsinghua University}\hfill\textbf{2007 - 2011}\\
Bachelor of Engineering in Computer Science\\
{Supervised by Andrew C. Yao}

{\Large\bf Employment}

\vspace{-5pt}
{\bf Senior Research Scientist at Semantic Machines, Inc}\hfill\textbf{2018 - Now}
\vspace{3pt}\\
{\bf Post-doc Researcher at Stanford University}\hfill\textbf{2016 - 2018}\\
{Hosted by Percy Liang and Moses Charikar}\vspace{3pt}\\
{\bf Intern at Baidu}\hfill\textbf{Winter, 2015}\\
{Project: burst detection in web search}\vspace{3pt}\\
{\bf Intern at Microsoft Research Redmond}\hfill\textbf{Summer, 2014}\\
{Project: convex optimization}\vspace{3pt}\\
{\bf Intern at Google Mountain View}\hfill\textbf{Summer, 2013}\\
{Project: personalized recommender systems}\vspace{3pt}\\
{\bf Intern at Microsoft Research Asia}\hfill\textbf{2010 - 2011}\\
{Project: click modeling for web search and online advertising}

{\Large\bf Selected Awards \& Honors}

\vspace{-5pt}
\textbf{2017}~~~~~~~~Best Paper Award, Conference on Learning Theory (COLT).\\
\textbf{2016}~~~~~~~~Outstanding Reviewer Award, International Conference on Machine Learning (ICML).\\
\textbf{2015}~~~~~~~~Baidu Fellowship (awards 8 PhD students every year worldwide).\\
\textbf{2013}~~~~~~~~Microsoft Research PhD Fellowship Finalist.\\
\textbf{2011}~~~~~~~~Outstanding Undergraduate Dissertation Award, Tsinghua University.\\
\textbf{2006}~~~~~~~~Silver Medal in Asian Physics Olympiad.\\
\textbf{2006}~~~~~~~~Gold Medal in Chinese Physics Olympiad ($5^{\rm{th}}$ among 400,000 participants).

{\bf\Large Research Projects}

My research interest lies in the field of \emph{artificial intelligence}. My past projects spanned machine learning, natural language processing and statistical methods. They consist of new algorithms, fundamental theory and practical software. Below is a list of projects that I have been working on (with publication references):

\vspace{-10pt}
\paragraph{Natural language processing}
\begin{my_item}
\item Semantic parsing for question answering systems [\ref{macro-emnlp17}].
\begin{my_item}
\item Efficient algorithm for parsing natural language into executable logical forms.
\item Impact: 11x-16x faster and 13\% more accurate than the state-of-the-art parser.
\end{my_item}
\end{my_item}

\vspace{-10pt}
\paragraph{Machine learning and optimization}
\begin{my_item}
\item Deep learning and non-convex optimization [\ref{convexified-icml17},\ref{a-hitting-colt17},\ref{on-the-learnability-aistats17},\ref{local-nips16},\ref{l1-icml16}].
\begin{my_item}
\item Algorithms for non-convex optimization (with applications to deep learning).
\item Impact: stronger theoretical guarantee and improved empirical performance on DNN/CNN/LSTM.
\end{my_item}
\item Convex optimization [\ref{stochastic-jmlr},\ref{stochastic-icml15}].
\begin{my_item}
\item Efficient algorithm for convex optimization.
\item Impact: orders-of-magnitude faster convergence than SGD in theory/practice.
\end{my_item}
\item Crowdsourcing [\ref{spectral-jmlr},\ref{spectral-nips14}].
\begin{my_item}
\item Algorithm for estimating the true labels from noisy crowdsourced data.
\item Impact: first algorithm to guarantee the best possible statistical accuracy.
\end{my_item}
\item Personalized recommender systems [\ref{taxonomy-wsdm14}].
\begin{my_item}
\item Non-parametric model for modeling very sparse user-item relations.
\item Impact: improved state-of-the-art models by 9\%-12\% on the quality of recommending less frequent items for online shopping.
\end{my_item}
\item Web search and online advertising [\ref{understanding-wsdm12},\ref{user-kdd11},\ref{characterize-www11},\ref{learning-cikm10},\ref{explore-cikm10},\ref{incorporating-sigir10}]
\begin{my_item}
\item Learning from massive search \& ads click logs to improve ranking quality.
\item Impact: currently in production as a part of Microsoft Bing. 
\end{my_item}
\end{my_item}

\vspace{-10pt}
\paragraph{Distributed computing}
\begin{my_item}
\item Distributed algorithms for machine learning [\ref{divide-jmlr},\ref{communication-jmlr},\ref{distributed-icml15},\ref{communication-icml15},\ref{divide-colt13},\ref{communication-nips12}].
\begin{my_item}
\item Communication-efficient algorithms for distributed machine learning on large clusters.
\item Impact: guarantee the best possible accuracy and the least possible computation/communication.
\end{my_item}
\item Foundamental theory of distributed computing [\ref{lower-colt14},\ref{information-nips13},\ref{optimality}].
\begin{my_item}
\item Understanding the foundamental trade-offs between communication, computation and statistical accuracy on any distributed system.
\item Impact: seminal papers on the study of communication complexity of distributed statistical estimation.
\end{my_item}
\item Programming interface for parallelizing stochastic algorithms [\ref{splash}].
\begin{my_item}
\item A user-friendly framework for parallelizing online algorithms on Spark.
\item Impact: component of Berkeley Data Analytics Stack (BDAS), 80+ Github stars.
\end{my_item}
\end{my_item}

\vspace{-10pt}
\paragraph{Other projects}
\begin{my_item}
\item Theoretical statistics [\ref{a-note-ejs},\ref{optimality-ejs},\ref{on-bayes-jmlr}].
\item Theoretical computer science [\ref{the-antimagicness-tcs},\ref{a-new-isaim10}].
\end{my_item}

{\bf\Large Manuscripts}
\vspace{5pt}

\begin{enumerate}[label={[M\arabic*]}, ref={M\arabic*}]
\item \label{splash}\textbf{Y. Zhang} and MI. Jordan. Splash: User-friendly Programming Interface for Parallelizing Stochastic Algorithms. \emph{arXiv:1506.07552}, 2015.

\item J. Duchi, MI. Jordan, M. Wainwright and \textbf{Y. Zhang} (alpha-beta order). Optimality Guarantees for Distributed Statistical Estimation. \emph{arXiv:1405.0782}, 2014.\label{optimality}
\end{enumerate}

{\bf\Large Journal Publications}
\vspace{5pt}

\begin{enumerate}[label={[J\arabic*]}, ref={J\arabic*}]

\item \textbf{Y. Zhang} and L. Xiao. Stochastic Primal-Dual Coordinate Method for Regularized Empirical Risk Minimization.
\emph{Journal of Machine Learning Research}.\label{stochastic-jmlr}

\item X. Chen, A. Guntuboyina and \textbf{Y. Zhang}. A note on the approximate admissibility of regularized estimators in the Gaussian sequence model.
\emph{Electronic Journal of Statistics}.\label{a-note-ejs}

\item \textbf{Y. Zhang}, M. Wainwright and MI. Jordan. Optimal prediction for sparse linear models? Lower bounds for coordinate-separable M-estimators.\label{optimality-ejs}
\emph{Electronic Journal of Statistics}.

\item X. Chen, A. Guntuboyina and \textbf{Y. Zhang} (alpha-beta order). On Bayes Risk Lower Bounds.
\emph{Journal of Machine Learning Research}.\label{on-bayes-jmlr}

\item \textbf{Y. Zhang}, X. Chen, D. Zhou and MI. Jordan. Spectral Methods meet EM: A Provably Optimal Algorithm for Crowdsourcing.
\emph{Journal of Machine Learning Research}.\label{spectral-jmlr}

\item \textbf{Y. Zhang}, J. Duchi and M. Wainwright. Divide and Conquer Kernel Ridge Regression: A Distributed Algorithm with Minimax Optimal Rates.
\emph{Journal of Machine Learning Research}.\label{divide-jmlr}

\item \textbf{Y. Zhang}, J. Duchi and M. Wainwright. Communication-Efficient Algorithms for Statistical Optimization.
\emph{Journal of Machine Learning Research}.\label{communication-jmlr}

\item \textbf{Y. Zhang} and X. Sun. The Antimagicness of the Cartesian Product of Graphs.
\emph{Theoretical Computer Science}.\label{the-antimagicness-tcs}
\end{enumerate}

{\bf\Large Conference Publications}
\vspace{5pt}

\begin{enumerate}[label={[C\arabic*]}, ref={C\arabic*}]

\item  \textbf{Y. Zhang}, P. Pasupat, P. Liang. Macro Grammars and Holistic Triggering for Efficient Semantic Parsing.
\emph{Empirical Methods on Natural Language Processing (EMNLP)}, 2017. \label{macro-emnlp17}

\item  \textbf{Y. Zhang}, P. Liang, M. Wainwright. Convexified Convolutional Neural Networks.
\emph{International Conference on Machine Learning (ICML)}, 2017. \label{convexified-icml17}

\item  \textbf{Y. Zhang}, P. Liang, M. Charikar. A Hitting Time Analysis of Stochastic Gradient Langevin Dynamics.
\emph{Conference on Learning Theory (COLT)}, 2017 {\bf (Best paper award)}. \label{a-hitting-colt17}

\item \textbf{Y. Zhang}, JD. Lee, M. Wainwright and MI. Jordan. On the Learnability of Fully-connected Neural Networks.
\emph{Artificial Intelligence and Statistics (AISTATS)}, 2017. \label{on-the-learnability-aistats17}

\item C. Jin, {\bf Y. Zhang}, S. Balakrishnan, M. Wainwright, MI. Jordan.  
Local Maxima in the Likelihood of Gaussian Mixture Models: Structural Results and Algorithmic Consequences.
\emph{Neural Information Processing Systems (NIPS)}, 2016. \label{local-nips16}

\item \textbf{Y. Zhang}, JD. Lee, MI. Jordan. $\ell_1$-regularized Neural Networks are Improperly Learnable in Polynomial Time.
\emph{International Conference on Machine Learning (ICML)}, 2016. \label{l1-icml16}

\item \textbf{Y. Zhang}, M. Wainwright and MI. Jordan. Distributed Estimation of Generalized Matrix Rank: Efficient Algorithms and Lower Bounds.
\emph{International Conference on Machine Learning (ICML)}, 2015. \label{distributed-icml15}

\item \textbf{Y. Zhang} and L. Xiao. DiSCO: Communication-Efficient Distributed Optimization of Self-Concordant Loss.
\emph{International Conference on Machine Learning (ICML)}, 2015. \label{communication-icml15}

\item \textbf{Y. Zhang} and L. Xiao. Stochastic Primal-Dual Coordinate Method for Regularized Empirical Risk Minimization.
\emph{International Conference on Machine Learning (ICML)}, 2015.\label{stochastic-icml15}

\item \textbf{Y. Zhang}, X. Chen, D. Zhou and MI. Jordan. Spectral Methods meet EM: A Provably Optimal Algorithm for Crowdsourcing.
\emph{Neural Information Processing Systems (NIPS)}, 2014. \textbf{(Spotlight presentation, 4.8\% acceptance rate)}
\label{spectral-nips14}

\item \textbf{Y. Zhang}, M. Wainwright and MI. Jordan. Lower Bounds on the Performance of Polynomial-time Algorithms for Sparse Linear Regression. \emph{Conference on Learning Theory (COLT)}, 2014. \label{lower-colt14}

\item \textbf{Y. Zhang}, A. Ahmed, V. Josifovski and A. Smola. Taxonomy Discovery for Personalized Recommendation. \emph{ACM International Conference on Web Search and Data Mining (WSDM)}, 2014. \label{taxonomy-wsdm14}

\item \textbf{Y. Zhang}, J. Duchi, M. Wainwright and MI. Jordan. Information-theoretic Lower Bounds for Distributed Statistical Estimation with Communication Constraints.
\emph{Neural Information Processing Systems (NIPS)}, 2013. \textbf{(Oral presentation, 1.4\% acceptance rate)}
\label{information-nips13}

\item  \textbf{Y. Zhang}, J. Duchi and M. Wainwright. Divide and Conquer Kernel Ridge Regression.
\emph{Conference on Learning Theory (COLT)}, 2013. \label{divide-colt13}

\item  \textbf{Y. Zhang}, J. Duchi and M. Wainwright. Communication-Efficient Algorithms for Statistical Optimization.
\emph{Neural Information Processing Systems (NIPS)}, 2012. \label{communication-nips12}

\item  W. Chen, D. Wang, \textbf{Y. Zhang} and Q. Yang. Understanding Click Noise: A Noise-aware Click Model for Web Search.
\emph{ACM International Conference on Web Search and Data Mining (WSDM)}, 2012. \label{understanding-wsdm12}

\item  \textbf{Y. Zhang}, W, Chen and D, Wang, Q. Yang. User-click Modeling for Understanding and Predicting Search-behavior.
\emph{ACM SIGKDD Conference on Knowledge Discovery and Data Mining (KDD)}, 2011.\label{user-kdd11}
\item B. Hu, \textbf{Y. Zhang}, G. Wang, Q. Yang, W. Chen. Characterize Search Intent Diversity into Click Models.
\emph{International World Wide Web Conference (WWW)}, 2011. \label{characterize-www11}

\item \textbf{Y. Zhang}, D. Wang, G. Wang, W. Chen, Z. Zhang, B. Hu and L. Zhang. Learning Click Model via Probit Bayesian Inference.
\emph{ACM International Conference on Information and Knowledge Management (CIKM)}, 2010. \label{learning-cikm10}

\item D. Wang, W. Chen, G. Wang, \textbf{Y Zhang} and B. Hu. Explore Click Models for Search Ranking.
\emph{ACM International Conference on Information and Knowledge Management (CIKM)}, short paper, 2010. \label{explore-cikm10}

\item F. Zhong, D. Wang, G. Wang, W. Chen, \textbf{Y. Zhang}, Z. Chen and H. Wang. Incorporating Post-Click Behaviors Into a Click Model.
\emph{Annual International ACM SIGIR Conference (SIGIR)}, 2010. \label{incorporating-sigir10}

\item \textbf{Y. Zhang} and L. Zhang. Extracting Independent Rules: a New Perspective of Boosting. 
\emph{International Symposium on Artificial Intelligence and Mathematics (ISAIM)}, 2010. \label{a-new-isaim10}
\end{enumerate}

{\Large\bf Teaching}

\vspace{-5pt}
Graduate Student Instructor, Introduction to machine learning, UC Berkeley \hfill\textbf{Spring, 2015}\\
Graduate Student Instructor, Randomized algorithms for matrices and data, UC Berkeley \hfill\textbf{Fall, 2013}

{\Large\bf Service}

\vspace{-5pt}
{\bf Journal Reviewer:} Journal of Machine Learning Research, Annals of Statistics, Mathematical Programming, ACM Transactions on the Web. \\
{\bf Conference Reviewer:} ICML (2013 - ), NIPS (2013 - ), AISTAT (2015 - ), IJCAI (2015 - ).

{\Large\bf Programming}

\vspace{-5pt}
Capable of Python, C/C++, C\#, Java, Scala, \textsc{Matlab}.

{\Large\bf References}

\begin{tabular}{ll}
\begin{minipage}{0.5\textwidth}
{\bf Michael I. Jordan}\\
Pehong Chen Distinguished Professor\\
EECS and Statistics, UC Berkeley\\
{\tt jordan@cs.berkeley.edu}
\end{minipage}&
\begin{minipage}{0.5\textwidth}
{\bf Martin J. Wainwright}\\
Professor\\
EECS and Statistics, UC Berkeley\\
{\tt wainwrig@eecs.berkeley.edu}
\end{minipage}\\\\
\begin{minipage}{0.5\textwidth}
{\bf Lin Xiao}\\
Principle Researcher\\
Machine Learning Department\\
Microsoft Research Redmond\\
{\tt lin.xiao@microsoft.com}
\end{minipage}&
\begin{minipage}{0.5\textwidth}
{\bf Percy Liang}\\
Assistant Professor\\
Computer Science Department\\
Stanford University\\
{\tt pliang@cs.stanford.edu}
\end{minipage}
\end{tabular}

\end{resume}
\end{document}
